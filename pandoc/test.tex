% {csquotes}
\chapter[\DTMusedate{entry}]{The Chapter Title}

This is an example file that is used to test the custom markdown to
LaTeX pandoc filter.

\section{Level 1 Section Header}

Some additional test text in multiple paragraphs: Some text, some
additional text as well as more text going on.

Here is the start of a new paragraph with some additional text alongside
with it.

\subsection{Level 2 Section Header}

Numbered list of items:

\begin{enumerate}
    \item Item number one potentially on multiple lines.

asdf

    \item Item number two with \emph{italics}

    \item Item number three with \textbf{bold}

\end{enumerate}

Bulleted list of items:

\begin{itemize}
\item
  First item
\item
  \emph{Second item}
\item
  Third item with potentially multiple lines

  Here is the additional line
\end{itemize}

\subsubsection{Level 3 Section Header}

Test for different text styles. Here is some \textbf{bold text}. Here is
some \textbf{italics} text.

\paragraph{Level 4 Section Header}

Test for quotations. Here is a text in quotes \enquote{testing one two
three.}

\begin{displayquote}
Quotes that are formatted like this. This is \textbf{BOLD}. This is
\emph{italics}
\end{displayquote}

And here are more quotes, that are now formatted like a block quote:

\begin{displayquote}
Quotes in a blockquote MULTILINE

that are on multiple lines

This is \textbf{BOLD}. This is \emph{italics}

going on like this.
\end{displayquote}

\section{Footnote tests}

Here is some text, and a footnote on this mark\footnote{Here is the
  content text of the aforementioned footnote. With some \textbf{bold}
  and some \textbf{italics} within it as well.}

A paragraph in between with some random filler content.

Additionally, this should be the example of a margin note using an
inline footnote in markdown. The inline footnote will begin on this
mark\footnote{Inline notes are easier to write, since you don't have to
  pick an identifier and move down to type the note.}

A paragraph in between with some random filler content.

\section{Code block tests}

\begin{listing}[h]
\begin{minted}[baselinestretch=1.0, frame=lines, mathescape, fontfamily=helvetica]{python}
# Comment
print("This is a code block in LaTeX")

print("Here is some more code")

def whitespace():
    x = 2
    return x
\end{minted}
\end{listing}